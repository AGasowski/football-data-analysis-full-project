\section{Apprentissage automatique}


L'apprentissage automatique est une sous-branche de l'intelligence
artificielle. Elle consiste à entraîner un modèle sur un jeu de données grâce à
un algorithme. Concrètement, au lieu de coder manuellement des règles pour
résoudre un problème, on fournit à l'ordinateur un grand nombre de données, à
partir desquelles il va reconnaître des motifs, faire des prédictions ou
prendre des décisions. L'apprentissage automatique est utilisé dans de nombreux
domaines : la reconnaissance d'images et de voix, les recommandations (comme
sur Netflix ou Spotify), les voitures autonomes, ou encore la détection de
fraudes bancaires.

\vskip 10pt

Ce type de technologie est également utilisé dans le domaine du sport. Dans le
cadre du football, il est possible d'exploiter les données issues des matchs,
des performances des joueurs ou encore des statistiques collectives pour tenter
d'anticiper certains résultats. À partir de toutes ces données, les fans de
football mais aussi les analystes essaient de prédire quelle équipe va faire
une très bonne saison et quelle équipe ne va pas réussir. C'est dans ce
contexte que nous nous posons la question suivante : comment peut-on prédire le
classement d'une équipe de football à l'aide de l'apprentissage automatique ?

\vskip 10pt

Nous avons utilisé deux modèles. Le premier modèle consiste à prédire le
classement d'une équipe à partir des saisons précédentes. Le deuxième modèle
consiste à prédire le classement final d'une équipe à partir des données des
matchs récoltées à la mi-saison.

\subsection{Premier modèle}

\subsubsection{Méthode linéaire}

Le premier modèle consistait à prédire le classement d'un championnat à partir
des caractéristiques des équipes et des saisons précédentes. Nous avons hésité
entre deux façons de faire. La première méthode consistait à utiliser les
caractéristiques de chaque équipe et à les classer à l'aide d'une régression
logistique. Le problème de ce modèle, c'est qu'il ne prenait pas en compte les
autres équipes du championnat : deux équipes ayant un niveau similaire
pouvaient ainsi être classées de la même manière. Pour les analystes, il n'est
pas nécessaire d'utiliser un modèle d'apprentissage automatique pour arriver à
ce genre de conclusion. Par exemple, on sait déjà que le FC Barcelone et le
Real Madrid seront parmi les premiers. Or, dans un classement, chaque position
est unique, ce qui rend cette approche limitée. 

\vskip 10pt

C'est pourquoi nous avons choisi une autre façon de faire. Nous avons opté pour
une approche locale : prédire les scores des matchs et ainsi en déduire un
classement précis. Pour prédire le score, nous avons utilisé le modèle de
régression linéaire multiple: nous avons choisi de manière assez subjective
plusieurs caractéristiques qui pouvaient influencer le score d'un match. Nous
réunissons les attributs dans un vecteur X. Nous utilisons la convention
suivante pour le nom des colonnes : "h" représente l'équipe à domicile (home)
et "a" désigne l'équipe à l'extérieur (away). Les principales colonnes
sélectionnées sont :

\begin{itemize}
    \item \texttt{moyenne\_overall\_titulaire\_h}
    \item \texttt{moyenne\_attributs\_buteur\_h}
    \item \texttt{moyenne\_attributs\_passeur\_h}
    \item \texttt{age\_moyen\_buteur\_h}
    \item \texttt{age\_moyen\_passeur\_h}
    \item \texttt{ratio\_but\_tir\_cadre\_h}
    \item \texttt{ratio\_tir\_cadre\_tir\_non\_cadre\_h}
    \item \texttt{y\_cards\_per\_match\_h}
    \item \texttt{r\_cards\_per\_match\_h}
    \item \texttt{nb\_unique\_contributors\_h}
    \item \texttt{buildUpPlaySpeed\_h}
    \item \texttt{buildUpPlayDribbling\_h}
    \item \texttt{buildUpPlayPassing\_h}
    \item \texttt{chanceCreationPassing\_h}
    \item \texttt{chanceCreationCrossing\_h}
    \item \texttt{defencePressure\_h}
    \item \texttt{defenceAggression\_h}
    \item \texttt{defenceTeamWidth\_h}
\end{itemize}

\vskip 10pt

La sortie Y est composée des variables \texttt{home\_team\_goal} et
\texttt{away\_team\_goal}. Nous avons ensuite séparé la base de données en deux
jeux de données : 80\% pour l'entraînement et 20\% pour le test. Comme nous
disposons de données sur 8 saisons, nous avons décidé d'utiliser les 6
premières saisons pour l'entraînement, et les 2 dernières pour le test. Nous
avons également standardisé les variables, afin d'éviter qu'une variable ait un
poids excessif ou insuffisant à cause de son échelle. Nous avons ensuite
entraîné notre modèle avec le module \texttt{LinearRegression().fit(X\_train,
Y\_train)}. Le coefficient R² obtenu est de 0.14, ce qui reste relativement
faible.

\vskip 10pt

Grâce à cet entraînement, nous avons pu tester notre modèle sur les saisons
2014/2015 et 2015/2016. Nous avons ainsi obtenu tous les scores prédits pour
ces saisons. À partir de ces scores, nous avons établi un classement, qui s'est
avéré relativement satisfaisant comparé au classement réel. Cependant, lorsque
des équipes créent la surprise, le modèle a du mal à les prédire, ce qui est
compréhensible. Par exemple, lors de la saison 2015/2016, Leicester a remporté
la Premier League de manière assez imprévisible dans le monde du football, le
modèle ne l'avait pas prévu

\vskip 10pt

La régression linéaire a des limites car elle utilisée pour des variables
continues or le score d'un match est une liste de deux entiers. Bien que nous
ayons des résultats satisfaisants, nous avons voulu comparer ce modèle avec un
modèle de régression de Poisson qui est un modèle plus adapté pour des sorties
entières.

\subsubsection{Méthode de Poisson}

Cette approche, bien que simple et intuitive, présente certaines limites liées
à la nature des données à prédire : les scores sont des comptages (0, 1, 2,
\dots), donc des variables discrètes et positives, tandis que la régression
linéaire repose sur l'hypothèse que la variable cible est continue et suit une
distribution gaussienne. Or, cette hypothèse est peu réaliste pour modéliser le
nombre de buts dans un match.

Quand on veut prédire une variable de comptage (comme un nombre de buts), on
peut supposer que cette variable suit une loi de Poisson. Cette distribution
est souvent utilisée pour modéliser le nombre d'événements survenant dans un
intervalle donné (un match, une étape, etc.), ce qui correspond bien à notre
cas : le nombre de buts au cours d'un match.

Formellement, on suppose que la variable cible suit une loi de Poisson :

\[
Y_i \sim \text{Poisson}(\lambda_i)
\]

où \(\lambda_i > 0\) est l'espérance du nombre de buts pour l'observation
\(i\), définie telle que :

\[
\lambda_i = \exp((X_i^T) \beta)
\]

où :

\begin{itemize}
    \item \(X_i \in \mathbb{R}^n\) est le vecteur des variables explicatives
    (par exemple : âge moyen des buteurs, ratio tirs cadrés / tirs, passes,
    etc.).
    \item \(\beta \in \mathbb{R}^n\) est le vecteur des coefficients à estimer.
\end{itemize}

L'usage de l'exponentielle garantit que \(\lambda_i\) reste strictement
positif, ce qui est nécessaire dans une loi de Poisson. En prenant le
logarithme, on obtient une relation linéaire entre \(\log(\lambda_i)\) et les
variables explicatives :

\[
\log(\lambda_i) = (X_i^T) \beta
\]

Un aspect fondamental du modèle de Poisson est que l'on suppose que la variance
de la variable cible conditionnellement aux variables explicatives est égale à
son espérance :

\[
\text{Var}(Y_i \mid X_i) = E(Y_i \mid X_i) = \lambda_i
\]

Autrement dit, dans un contexte de buts pour une observation donnée, plus
l'incertitude autour de cet événement est grande, c'est l'hypothèse
d'équidispersion.

Elle permet de simplifier les calculs d'estimation et rend les intervalles de
confiance relativement serrés, mais elle peut être mise en défaut dans de
nombreux cas réels. En revanche, si les données sont sur-dispersées (variance >
moyenne), cela peut biaiser les résultats, et c'est bien une limite de ce
modèle dans certaines situations.

\vskip 10pt

Concrètement, l'apprentissage consiste à estimer le vecteur \(\beta\) en
maximisant la vraisemblance du modèle, c'est-à-dire la probabilité conjointe
d'observer les données \(Y_1, \dots, Y_n\) conditionnellement aux variables
explicatives \(X_1, \dots, X_n\) sous nos hypothèses.

\vskip 10pt

\textbf{À noter :} lorsque \(\lambda\) est assez grand (typiquement \(>10\)),
la loi de Poisson devient plus symétrique et tend vers une loi normale
(théorème de la limite centrale). Dans ce cas, une régression linéaire peut
redevenir pertinente. Mais dans notre situation, comme \(\lambda\) représente
le nombre moyen de buts marqués par une équipe dans un match, il reste souvent
inférieur à 2. Cela justifie le recours à un modèle spécifique aux variables
discrètes et rares.

\medskip

Enfin, plutôt que de choisir arbitrairement l'un des deux modèles, nous avons
conservé à la fois la régression linéaire multiple et la régression de Poisson
afin de les comparer objectivement. Nous avons utilisé pour cela l'indicateur
RMSE (Root Mean Square Error), qui mesure l'écart quadratique moyen entre les
scores prédits et les scores observés.

\begin{figure}[H]
    \centering
    \includegraphics[width=0.8\textwidth, keepaspectratio]{Images/lin_meth1.png}
    \begin{centering}
        \captionsetup{justification=centering}
    \caption{Prédiction au début de saison du classement final de Villareal
    pour la saison 2014/2015 (Méthode linéaire)}
    \end{centering}
\end{figure}

\vskip 10pt

\begin{figure}[H]
    \centering
    \includegraphics[width=0.8\textwidth, keepaspectratio]{Images/poi_meth1.png}
    \begin{centering}
        \captionsetup{justification=centering}
    \caption{Prédiction au début de saison du classement final de Villareal
    pour la saison 2014/2015 (Méthode de Poisson)}
    \end{centering}
\end{figure}

\vskip 10pt

Même si l'écart de RMSE est minime, cela justifie d'avoir testé cette approche
: les scores sont discrets, donc un modèle conçu pour les variables de comptage
est statistiquement plus cohérent. On observe aussi un écart de 0.86 dans
l'intervalle de confiance de la régression multiple face à 0.82 pour Poisson.
En ce sens, cela montre que le modèle de régression de Poisson est plus fiable
dans sa prédiction.

\vskip 10pt

\subsection{Deuxième modèle}

Le deuxième modèle consiste à prédire le classement d'une équipe à la fin du
championnat grâce aux matchs joués lors de la première moitié de saison. Nous
avons collecté les informations essentielles de la mi-saison pour prédire le
classement final. Pour cela, nous avons à nouveau utilisé un modèle de
régression de Poisson. Les caractéristiques que nous avons choisies sont :

\begin{itemize}
    \item \texttt{points},
    \item \texttt{goal\_diff},
    \item \texttt{buts\_marques},
    \item \texttt{buts\_encaisses},
    \item \texttt{victoires},
    \item \texttt{nuls},
    \item \texttt{defaites},
    \item \texttt{forme\_5\_derniers},
    \item \texttt{matchs\_dom},
    \item \texttt{matchs\_ext}.
\end{itemize}

Voici le résultat obtenu lors de la saison 2014/2015 pour l'équipe de Villareal
est :

\vskip 10pt

\begin{figure}[H]
    \centering
    \includegraphics[width=0.8\textwidth, keepaspectratio]{Images/poi_meth2.png}
    \begin{centering}
        \captionsetup{justification=centering}
    \caption{Prédiction à mi-saison du classement final de Villareal pour la
    saison 2014/2015 (Méthode de Poisson)}
    \end{centering}
\end{figure}

Cette deuxième approche permet d'obtenir un autre résultat, qui dans ce cas
précis se rapproche plus de la réalité. En effet, Villareal avait fini 5e de
son championnat cette saison-là.

\vskip 10pt

En conclusion, les différents modèles nous ont permis d'explorer plusieurs
approches pour prédire le classement d'un championnat. Chaque modèle a ses
points forts et ses limites. Il serait intéressant d'ajouter davantage de
données afin de mieux anticiper des résultats futurs, même si le football reste
imprévisible et que nos modèles d'apprentissage ne sont pas encore assez
sophistiqués pour offrir des prédictions très précises.