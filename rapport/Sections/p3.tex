\section{Apprentissage automatique}


Notre travail consiste à prédire le classement final d'un championnat à l'aide
des caractéristiques des équipes, des joueurs, ainsi que des caractéristiques
des matchs de chaque équipe. Pour cela, nous utilisons une régression linéaire
multiple. En entrée, nous avons une table que l'on peut nommer Match, et en
sortie, nous avons la liste des scores que nous souhaitons prédire. Grâce à ces
résultats on pourra en déduire le classement du championnat.

\subsection{Préparation de la table Match}

Tout d'abord, il faut préparer la table Match. Nous utilisons la convention
suivante pour le nom des colonnes: "h" représente l'équipe à domicile (home) et
"a" désigne l'équipe à l'extérieur (away). Les principales colonnes
sélectionnées sont:
\begin{itemize}
    \item "moyenne\_attributs\_buteur\_h"
    \item "moyenne\_attributs\_passeur\_h"
    \item "age\_moyen\_buteur\_h"
    \item "age\_moyen\_passeur\_h"
    \item "ratio\_but\_tir\_cadre\_h"
    \item "ratio\_tir\_cadre\_tir\_non\_cadre\_h"
    \item "y\_cards\_per\_match\_h"
    \item "r\_cards\_per\_match\_h"
    \item "nb\_unique\_contributors\_h"
    \item "buildUpPlaySpeed\_h"
    \item "buildUpPlayDribbling\_h"
    \item "buildUpPlayPassing\_h"
    \item "chanceCreationPassing\_h"
    \item "chanceCreationCrossing\_h"
    \item "defencePressure\_h"
    \item "defenceAggression\_h"
    \item "defenceTeamWidth\_h"
\end{itemize}

À l'origine, nous voulions calculer la moyenne des attributs de tous les
joueurs d'une équipe, mais la structure de la base de données ne nous permet
pas de lier directement les joueurs à leurs équipes. Nous avons donc opté pour
la moyenne des attributs des buteurs et des passeurs, en considérant qu'ils
contribuent le plus à la performance de l'équipe. En résumé, toutes les
variables listées ci-dessus nous ont semblé suffisamment pertinentes. Le but,
par la suite, est aussi d'affiner la sélection des variables en ajoutant ou en
supprimant celles qui pourraient être nuisibles à la performance du modèle.

\subsection{Traitement des attributs d'équipe}

Nous avons rencontré quelques difficultés avec la table Team\_Attributes, qui
contient les attributs d'une équipe à une date donnée. Il arrive qu'une même
équipe apparaisse plusieurs fois au cours d'une même saison. Dans ce cas, nous
avons choisi de conserver la ligne présentant les attributs les plus élevés.
Cependant, pour certaines équipes, aucun attribut n'était disponible pour une
saison donnée. Dans ces cas-là, nous avons utilisé les attributs de la saison
précédente. Nous avons rencontré le même problème avec la table
Player\_Attributes, et avons adopté la même méthode. 

\subsection{Modèle et prédiction}

Finalement, nous avons réussi à construire la table Match, qui constitue notre
entrée X, et la sortie Y est composée des variables home\_team\_goal et
away\_team\_goal. Nous avons ensuite séparé la base de données en deux jeux de
données : 80\% pour l'entraînement et 20\% pour le test. Comme nous disposons
de données sur 8 saisons, nous avons décidé d'utiliser les 6 premières saisons
pour l'entraînement, et les 2 dernières pour le test. Nous avons également
standardisé les variables, afin d'éviter qu'une variable ait un poids excessif
ou insuffisant à cause de son échelle. Nous avons ensuite entraîné notre modèle
avec le module LinearRegression().fit(X\_train, Y\_train). Le coefficient R²
obtenu est de 0,14, ce qui reste relativement faible.

\vskip 10pt

Grâce à cet entraînement, nous avons pu tester notre modèle sur les saisons
2014/2015 et 2015/2016. Nous avons ainsi obtenu tous les scores prédits pour
ces saisons. À partir de ces scores, nous avons établi un classement, qui s'est
avéré relativement satisfaisant comparé au classement réel. Cependant, lorsque
des équipes créent la surprise, le modèle a du mal à les prédire, ce qui est
compréhensible. Par exemple, lors de la saison où Leicester a remporté la
Premier League, le modèle ne l'avait pas prévu.

\subsection{Perspectives}

Par la suite, nous aimerions tenter de prédire le classement d'un championnat à
partir des premières journées de la saison. Il faudra donc revoir la façon dont
nous constituons X\_train et X\_test.