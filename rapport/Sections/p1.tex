\section{Présentation du jeu de données}

\vskip 10pt

Pour réaliser notre travail, nous avons utilisé une base de données
relationnelle composée de plusieurs tables au format CSV. Celle-ci regroupe des
données sur le football européen, allant de 2008 à 2016. Par souci de clarté,
nous n'avons pas listé l'ensemble des champs de chaque table.

\begin{figure}[H]
    \centering
    \includegraphics[width=1.0\textwidth, keepaspectratio]{Images/schema_relationnel2.png}
    \begin{centering}
        \captionsetup{justification=centering}
    \caption{Schéma relationnel \\
    \textit{Note de lecture : Une équipe peut jouer plusieurs matchs
    (cardinalité 1,N), mais chaque match ne peut avoir qu'une seule équipe à
    domicile (cardinalité 1,1).} \\
    \textit{Réalisé sur dbdiagram.io}}
    \end{centering}
\end{figure}

\vskip 10pt

Les différentes tables qui composent notre base de données sont : 

\vskip 10pt

\begin{itemize}
    \item \textbf{Country} : Table de référence listant les pays. Elle associe
    un identifiant unique à chaque pays, facilitant ainsi les relations avec
    d'autres tables.
    \item \textbf{League} : Contient les championnats, en lien avec un pays.
    Chaque ligue a un nom et une clé étrangère vers \textit{Country}.
    \item \textbf{Match} : Table centrale regroupant les rencontres entre
    équipes, avec les scores, les identifiants des équipes en jeu et de
    nombreuses statistiques.
    \item \textbf{Player} : Contient les informations de base sur les joueurs,
    telles que le nom, la taille et le poids.
    \item \textbf{Player\_Attributes} : Données dynamiques décrivant les
    caractéristiques des joueurs à différentes dates (note générale, pied
    préféré, etc.).
    \item \textbf{Team} : Donne des informations statiques sur les équipes (nom
    complet, nom abrégé).
    \item \textbf{Team\_Attributes} : Caractéristiques des équipes évoluant
    dans le temps (style de jeu, tactiques, etc.).
    \item \textbf{Goal} : La table Goal n’existe pas en tant que table
    distincte dans le fichier CSV, car les données relatives aux buts sont
    imbriquées dans un champ XML de la table principale Match. Plus
    précisément, la colonne goal contient, pour chaque match, un bloc XML
    décrivant les événements de but (joueur, minute, type de but, équipe,
    etc.). Ici, nous ne représentons que la table Goal, mais d'autres tables de
    même nature existent également, comme Shoton, Shotoff, Foulcommit, Card,
    Cross, Corner, et Possession.
\end{itemize}

\vskip 10pt
