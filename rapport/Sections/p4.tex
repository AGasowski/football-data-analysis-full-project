\section{Caractéristiques techniques}

\subsection{Architecture du projet}

L'architecture de notre projet repose sur une séparation claire des
responsabilités à travers plusieurs modules \texttt{Python}. Chaque question à
laquelle nous répondons est traitée dans un fichier spécifique, ce qui favorise
la modularité et la réutilisabilité du code. Les fonctions communes, telles que
le chargement de données ou les opérations de filtrage, sont centralisées dans
des modules partagés. L'utilisateur n'a besoin de lancer que le fichier
\_\_main\_\_.py, qui initialise l'interface via la classe AccueilView. Cette
interface interactive permet de sélectionner dynamiquement les questions à
exécuter sans avoir à modifier manuellement le code, assurant ainsi une
expérience utilisateur fluide et intuitive.

\subsection{Bibliothèques utilisées}

Dans ce projet, nous avons utilisé plusieurs bibliothèques pour faciliter la
manipulation des données, la visualisation et l'interaction avec l'utilisateur.
La bibliothèque \texttt{pandas} a été essentielle pour le traitement des
données tabulaires. Elle s'est révélée bien plus simple et puissante que
l'utilisation de modules comme \texttt{os} pour manipuler manuellement des
fichiers ou des structures de données. Pour la visualisation, nous avons
parfois utilisé \texttt{matplotlib}, qui nous a permis de créer des graphiques
personnalisés, notamment en utilisant des objets comme \texttt{patches} pour
enrichir l'affichage. D'autres modules standards comme \texttt{csv},
\texttt{datetime}, \texttt{collections} ou \texttt{pathlib} ont été employés
pour diverses tâches techniques, comme le parsing, la gestion des chemins, ou
le traitement de dates. Enfin, pour l'affichage d'une interface utilisateur en
ligne de commande, nous avons utilisé \texttt{InquirerPy}, qui offre une
navigation interactive agréable et dynamique.

\subsection{Tests}

Afin de garantir la fiabilité et la robustesse de notre code, nous avons mis en
place une suite de tests automatisés à l'aide du framework \texttt{pytest}. Les
résultats obtenus montrent un taux de couverture global de COMPLETER A LA FIN,
ce qui reflète un bon niveau de validation du code. Nous avons également
souhaité valider la cohérence de notre implémentation en vérifiant que le
classement de Premier League calculé à l'aide de notre code correspond
exactement au classement officiel. Ces résultats traduisent un travail sérieux
de tests, qui n'étaient pas au centre du projet, tout en mettant en lumière
certaines zones à renforcer pour atteindre une couverture plus homogène et
optimale sur l'ensemble du projet.

METTRE L'IMAGE DE LA COURVETURE QUAND TESTS FINIS


\subsection{Documentation et commentaires}

Une attention particulière a été mise sur l'écriture de la documentation. Cela
nous semblait essentiel dans un projet de groupe, où l'organisation et
l'utilisation du code par différents membres nécessitent une compréhension
claire des fonctions, des structures de données et des choix techniques
effectués.



