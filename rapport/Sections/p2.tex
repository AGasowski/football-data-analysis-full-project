\section{Présentation des questions posées}

\subsection{Questions imposées}

Deux questions nous étaient imposées par le sujet. Nous avons décidé d'ajouter
des fonctionnalités pour que l'utilisateur puisse choisir des paramètres de ces
questions.

\vskip 10pt

\textbf{Question 1 : Quel est le classement de la Premier League lors de la
saison 2015/2016 ?}

\vskip 10pt

Pour réaliser cette question, nous avons codé en Python pur, c'est-à-dire sans
l'aide du module pandas. Nous avons utilisé les règles officielles pour le
classement de la Premier League. Les critères, par ordre d'importance, sont les
suivants: le nombre de points, puis la différence de buts, puis le nombre de
buts marqués. Nous avons voulu laisser la possibilité à l'utilisateur de
choisir une autre saison que 2015/2016. De plus, il a la possibilité de
demander le classement d'un autre championnat. Ce dernier point doit être
précisé, puisque les critères utilisés pour départager deux équipes ayant le
même nombre de points dépend du championnat. En France par exemple, si la
différence de buts est la même, nous regardons les points obtenus lors des
confrontations directes entre les deux équipes. Cependant, dans un souci de
temps imparti, nous n'avons pas pu ajuster les critères en fonction des
championnats et les classements sont réalisés selon les règles de la Premier
League.

\vskip 10pt

\textbf{Question 2 : Quels sont les matchs avec la plus grande différence de
buts dans le jeu de données ?}

\vskip 10pt

Nous avons répondu à la question en utilisant le paquet pandas. De plus, nous
avons ajouté la possibilité pour l'utilisateur de choisir s'il veut obtenir les
matchs avec la plus grande différence de buts pour toutes les saisons
confondues, ou bien pour une saison bien particulière.

\subsection{Questions choisies}

\textbf{Question 3 : Quelle est la taille moyenne des joueurs ayant marqué de
la tête ?}

\vskip 10pt

Pour trouver l'intitulé de cette question, nous voulions profiter de la liberté
que nous possédions afin de déterminer une statistique descriptive que l'on ne
trouve pas habituellement. La taille moyenne des joueurs marquant de la tête
peut sembler anecdotique à première vue, mais nous souhaitions observer si
cette taille était élevée, ou bien similaire à la taille moyenne d'une
population de joueurs de foot. Comme pour les questions précédentes, nous avons
permis à l'utilisateur de choisir une saison en particulier ou non. De plus,
nous l'avons traitée avec pandas.

\vskip 10pt

\textbf{Question 4 : Quel est le classement, tous championnats confondus, des
cartons jaunes, des cartons rouges, des buteurs et des passeurs ?}

\vskip 10pt

Ces quatre questions sont en réalité très similaires, d'où notre volonté de les
rassembler en une seule question. Ce sont des statistiques très facilement
accessibles sur les sites spécialisés puisqu'elles sont au cœur des débats
footballistiques. Elles ont toutes été traitées avec le paquet pandas.
L'utilisateur peut choisir la saison pour laquelle il veut afficher le
classement en question.

\vskip 10pt

\textbf{Question 5 : Combien de fois chaque dispositif a-t-il été utilisé ?}

\vskip 10pt

Au départ, nous voulions connaître le dispositif le plus utilisé, et qui semble
être le préféré des entraîneurs. La réponse renvoyée était la même pour chaque
saison: 4 2 3 1. Nous avons alors eu envie de connaître les autres dispositifs
les plus utilisés, d'où la création de ce classement. Il est important de
rappeler que la base de données ne prend pas en compte les éventuels
changements de dispositif au cours d'un match. La réalisation de cette question
a été faite en utilisant pandas.

\vskip 10pt

\textbf{Question 6 : Quel est le jour, entre 2008 et 2016, qui a connu le plus
de matchs nuls ?}

\vskip 10pt

De nouveau, nous avions envie de chercher un résultat original. Pour cela, nous
avons utilisé le paquet pandas. Nous cherchions à identifier un jour
particulièrement notable, et le jour ayant connu le plus de matchs nuls était
idéal. Il s'agit d'un samedi, ce qui semble logique puisque plus de matchs y
sont joués. Le paquet pandas nous a de nouveau servi ici.

\vskip 10pt

\textbf{Question 7 : Quelle est l'équipe type en 4 4 2 ?}

\vskip 10pt

Chaque weekend, les médias de foot déterminent leur équipe type. La sélection
et la comparaison des joueurs est presque une obsession pour les fans de foot,
d'où notre volonté de créer notre propre sélection. Tout d'abord, la
détermination des critères d'une équipe type est évidemment subjective.
Ensuite, nous avons choisi un dispositif précis et populaire (le 4 4 2) afin de
simplifier le choix des critères en fonction des postes. La méthode utilisée
pour sélectionner un joueur à un poste est la suivante:

\begin{itemize}
    \item \textbf{Sélection des critères les plus importants pour jouer à un
    poste.} Par exemple, en ce qui concerne les défenseurs centraux, nous avons
    sélectionné des aptitudes telles que l'agressivité, l'interception, les
    tacles, la détente, la force et le marquage. Nous ne développons pas les
    aptitudes sélectionnées pour chaque poste, mais la sélection a été faite en
    nous renseignant sur celles qui étaient les plus importantes sur des médias
    de tactique de foot.
    \item \textbf{Calcul des scores.} Le calcul du score est réalisé en faisant
    la somme des scores de chaque critère sélectionné pour le poste. En effet,
    chaque joueur possède une note pour chaque aptitude. Nous choisissons
    ensuite le joueur qui obtient le meilleur score et qui n'est pas déjà
    présent dans notre équipe type. 
    \item \textbf{Choix du pied fort.} Sur un terrain, il est plus probable de
    jouer à droite en étant droitier, et inversement en étant gaucher. Pour
    retrouver une certaine cohérence avec la réalité, nous voulions
    sélectionner le côté des joueurs en fonction de leur pied fort. Pour cela,
    les postes de latéral, d'ailier et de défenseur central sont concernés.
\end{itemize}

\vskip 10pt

L'utilisateur peut ici choisir la saison pour laquelle il souhaite obtenir
l'équipe type. Pour une visualisation plus agréable, nous avons modélisé un
beau terrain de foot afin de placer les joueurs au bon endroit. Toute cette
question a été réalisée à l'aide du paquet pandas. 

\vskip 10pt

\textbf{Question 8 : Quelles sont les équipes qui sont les plus régulières en
terme de performances ?}

Il est intéressant de savoir si certaines équipes sont constantes dans leurs
performances, que ça soit avec de bonnes ou moins bonnes aptitudes. Pour
répondre à cette question, il est possible de proposer de nombreuses méthodes.
Nous allons donc détailler la nôtre. Nous utilisons d'abord la table Matchs
afin de connaître les 11 joueurs les plus souvent alignés pour chaque équipe et
simplifier notre démarche. Nous utilisons ensuite la table des attributs des
joueurs afin d'évaluer l'écart-type de ses attributs techniques à différents
instants.

Soit \( E \) un ensemble d'équipes, et pour chaque équipe \( e \in E \), soit
\( J_e = \{j_1, j_2, \dots, j_{11}\} \) l'ensemble des 11 joueurs les plus
souvent alignés.  
Pour chaque joueur \( j \in J_e \), on considère un vecteur d'attributs
techniques au cours du temps, noté \( A_j(t) \), et on calcule :

\[
\sigma_j = \frac{1}{n} \sum_{k=1}^{n} \text{écart-type}(A_j(t_k))
\]

où \( t_1, t_2, \dots, t_n \) sont les instants d'observation.

Ensuite, pour chaque équipe \( e \), on calcule la moyenne des écarts-types des
11 joueurs :

\[
\Sigma_e = \frac{1}{11} \sum_{j \in J_e} \sigma_j
\]

Une faible valeur de \( \Sigma_e \) indique que les joueurs de l’équipe \( e \)
présentent des performances cohérentes dans le temps. Nous obtenons ainsi un
classement des équipes selon ce critère.

\vskip 10pt

\textbf{Question 9 : Quelle équipe marque, en moyenne, plus à l'extérieur qu'à
domicile ?}

\vskip 10pt

Jouer à domicile est souvent considéré comme un avantage. Les supporters sont
parfois appelés le 12e homme, les joueurs sont familiers avec l'environnement,
ils n'ont pas besoin d'effectuer un long trajet. Tout semble en effet favoriser
l'équipe qui joue à domicile. Cependant, certaines équipes ont de meilleurs
résultats à l'extérieur, et c'est ce que nous avons voulu mettre en avant dans
cette question. Nous avons réalisé le classement en fonction de l'écart entre
le nombre moyen de buts marqués à l'extérieur et à domicile. Nous avons étudié
l'écart des moyennes pour que la comparaison entre les équipes soit pertinente.
Cette fois, nous avons répondu à la question en Python pur.

\vskip 10pt

\textbf{Question 10 : Quelle est l'évolution des attributs physiques des
joueurs en fonction de leur âge ?}

\vskip 10pt

Il est intéressant d'étudier l'évolution de certaines capacités physiques en
fonction de l'âge des joueurs. Nous pensons que l'accélération, la force et
l'endurance sont trois facteurs qui varient fortement en fonction de l'âge,
d'où notre choix d'étude de ceux-ci. Nous calculons ensuite la moyenne des
attributs sélectionnés pour chaque catégorie d'âge. Le résultat de cette
question est un graphique affichant clairement la moyenne pour chaque
catégorie. On y voit facilement l'évolution des attributs, ce qui est plutôt
intéressant à observer.

\vskip 10pt

\textbf{Question 11 : Quelle équipe a eu le pire ratio buts marqués / tirs
cadrés ?}

\vskip 10pt

L'idée était de connaître l'équipe avec le moins de réussite, par rapport à ses
frappes cadrées. En d'autres termes, quelle équipe a dû faire face à des
gardiens très efficaces ? L'utilisateur a de nouveau la possibilité de choisir
la saison pour laquelle il souhaite consulter cette statistique. Cette question
a été réalisée en Python pur.

\vskip 25pt 

Nous avons souhaité offrir des résultats qui allient nos connaissances
statistiques, des choix subjectifs dans la création et la sélection des
résultats, ainsi que de l’originalité. L’accent a été mis sur l’interactivité
de l’utilisateur et sur une visualisation agréable des résultats, tout en
utilisant diverses techniques pour enrichir l'expérience.