\section{Présentation des questions posées}

\subsection{Questions imposées}

Deux questions nous étaient imposées par le sujet. Nous avons décidé d'ajouter
des fonctionnalités pour que l'utilisateur puisse choisir des paramètres de ces
questions.


\textbf{Question 1 :} Quel est le classement de la Premier League lors de la \\
saison 2015/2016 ? 


Pour réaliser cette question, nous avons codé en python pur, c'est-à-dire sans
l'aide du module pandas. Nous avons utilisé les règles officielles pour le
classement de la Premier League. Les critères, par ordre d'importance, sont les
suivants: le nombre de points, la différence de buts et le nombre de buts
marqués. Nous avons voulu laisser la possibilité à l'utilisateur de choisir une
autre saison que 2015/2016. De plus, il a la possibilité de demander le
classement d'un autre championnat. Ce dernier point doit être précisé, puisque
les critères utilisés pour départager deux équipes ayant le même nombre de
point dépend du championnat. En France par exemple, si la différence de buts
est la même, nous regardons les points obtenus lors des confrontations directes
entre les deux équipes. Cependant, dans un souci de temps imparti, nous n'avons
pas pu ajuster les critères en fonction des championnats et les classements
sont réalisés selon les règles de la Premier League.


\textbf{Question 2 :} Quels sont les matchs avec la plus grande différence \\
de buts dans le jeu de données ?


Cette fois, nous avons répondu à la question en utilisant le paquet pandas. De
plus, nous avons ajouter la possibilité pour l'utilisateur de choisir si il
veut obtenir les matchs avec la plus grande différence de buts pour toutes les
saisons confondues, ou bien pour une saison bien particulière.


\subsection{Questions choisies}


\textbf{Question 3 :} Quels sont les matchs avec la plus grande différence \\
de buts dans le jeu de données ?