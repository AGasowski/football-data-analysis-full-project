\addcontentsline{toc}{section}{Introduction}

\section*{Introduction}

Le football fait partie des sports les plus populaires au monde. Il rassemble
les sportifs professionnels, les fans et les amateurs autour d'une passion
commune. Les nombreux médias (chaînes de télévision, émissions spécialisées,
comptes sur les réseaux sociaux, journaux, etc\ldots{}) prennent une part
importante de l'espace public en France.

\vspace{1em}

Le monde du football a beaucoup évolué, notamment dans son utilisation des
technologies et des données. Par exemple, les capacités des joueurs sont
scrutées et analysées grâce à des capteurs. Les statistiques des matchs sont de
plus en plus précises avec notamment les calculs d'XG (expected goals, buts
attendus en fraçais). Les paris sportifs se sont largement développés grâce aux
larges bases de données mises à disposition des bookmakers.

\vspace{1em}

C'est dans ce contexte que nous avons eu envie de découvrir et de travailler
nous-mêmes sur des données liées au football, afin de répondre à certaines
questions classiques, mais également des questions plus spécifiques et
originales.